% This file provides an example Beamer presentation using the RWTH theme
% showcasing some of the more common options, similar to the Powerpoint version
% 12.11.2014: Revision 1 (Harold Bruintjes, Tim Lange)

% For RWTH, beamer should be loaded with class option t (top)
\documentclass[t]{beamer}

% Use fontspec to get Arial font
% Requires use of XeLaTeX
\usepackage{fontspec}
\setmainfont{Arial}
\setsansfont{Arial}
% Also force Arial for math for a more consistent look
\usepackage{unicode-math}

% German style date formatting (footer)
\usepackage[ddmmyyyy]{datetime}
\renewcommand{\dateseparator}{.}

\usepackage{MnSymbol,wasysym}

% Format the captions used for figures etc.
\usepackage[compatibility=false]{caption}
\captionsetup{singlelinecheck=off,justification=raggedleft,labelformat=empty,labelsep=none}

% PGFPlots is used for drawing some of the charts
\usepackage{pgfplots}
\pgfplotsset{compat=newest}
\input{plot_commands.tex}

% Load the actual RWTH theme. Suggested is to load the full theme,
% as it requires some specific dimensions
\usetheme{rwth}

\begin{document}

\logo{\includegraphics{logo.png}}

% Setup presentation information
\title{Syntaktische Mehrdeutigkeiten \\ Erkennung, Vermeidung und Auflösung}
\date{07.06.2024}
\author{Lennart Protte}

\frame{\titlepage}


\section{Motivation}
\begin{frame}
\begin{itemize}
    \item Bedeutung von syntaktischen und lexikalischen Mehrdeutigkeiten
    \item Auswirkungen auf die Zuverlässigkeit und Effizienz von Parsern und Compilern
    \item Ziel: Erkennen, Vermeiden und Lösen von Mehrdeutigkeiten
\end{itemize}
\end{frame}

\section{Inhalt}
\begin{frame}
\tableofcontents
\end{frame}

\section{Probleme und Herausforderungen}\label{sec:probleme-und-herausforderungen}
\begin{frame}
\begin{itemize}
    \item Definition syntaktischer und lexikalischer Mehrdeutigkeiten
    \item Beispiele für Mehrdeutigkeiten in Programmiersprachen
    \item Negative Konsequenzen für die Parserleistung
\end{itemize}
\end{frame}

\section{Theoretische Ergebnisse und algorithmische Lösungen}\label{sec:theoretische-ergebnisse-und-algorithmische-losungen}
\begin{frame}
\begin{itemize}
    \item Formale Sprachen und Grammatiken
    \item Algorithmen zur Erkennung von Mehrdeutigkeiten (Lookahead, Parse Trees)
    \item Lösungsstrategien: Vorrangsregeln, Assoziativität, Grammatik-Korrektur
\end{itemize}
\end{frame}

\section{Praktische Ansätze}\label{sec:praktische-ansatze}
\begin{frame}
\begin{itemize}
    \item Beispiele aus der Praxis: Bekannte Compiler und Programmiersprachen
    \item Fallstudien: Anwendung der Techniken in realen Projekten
\end{itemize}
\end{frame}

\section{Ausblick und zukünftige Projekte}\label{sec:ausblick-und-zukunftige-projekte}
\begin{frame}
\begin{itemize}
    \item Potenzielle Weiterentwicklungen: Neue Herausforderungen und Fragen
    \item Ideen für Folgeprojekte: Erweiterung auf neue Sprachfamilien, maschinelles Lernen zur Erkennung von Mehrdeutigkeiten
\end{itemize}
\end{frame}

\section{Quellen}\label{sec:quellen}
\begin{frame}
\begin{itemize}
    \item Referenzen und weiterführende Literatur
    \item Verweise auf den vollständigen Bericht und zusätzliche Materialien
\end{itemize}
\end{frame}

\end{document}