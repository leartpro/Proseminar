% This file provides an example Beamer presentation using the RWTH theme
% showcasing some of the more common options, similar to the Powerpoint version
% 12.11.2014: Revision 1 (Harold Bruintjes, Tim Lange)

% For RWTH, beamer should be loaded with class option t (top)
\documentclass[t]{beamer}

% Use fontspec to get Arial font
% Requires use of XeLaTeX
\usepackage{fontspec}
\setmainfont{Arial}
\setsansfont{Arial}
% Also force Arial for math for a more consistent look
\usepackage{unicode-math}

% German style date formatting (footer)
\usepackage[ddmmyyyy]{datetime}
\renewcommand{\dateseparator}{.}

\usepackage{MnSymbol,wasysym}

% Format the captions used for figures etc.
\usepackage[compatibility=false]{caption}
\captionsetup{singlelinecheck=off,justification=raggedleft,labelformat=empty,labelsep=none}

% PGFPlots is used for drawing some of the charts
\usepackage{pgfplots}
\pgfplotsset{compat=newest}
\input{plot_commands.tex}


%custom
\usepackage{tikz}
\usetikzlibrary{trees}
\usetikzlibrary{automata, positioning}
\usepackage{listings}
\lstset{basicstyle=\ttfamily}
\usepackage{tcolorbox}
\usepackage{framed}
\usepackage{forest}
\usepackage{xcolor}
\usepackage{blkarray}
\usepackage{comment}

% Define C++ style
\lstdefinestyle{cppStyle}{
	language=C++,
	basicstyle=\Large\ttfamily,
	keywordstyle=\color{blue},
	stringstyle=\color{purple},
	commentstyle=\color{green!60!black},
	morecomment=[l][\color{magenta}]{\#},
	tabsize=2,
	breaklines=true,
	showspaces=false,
	showstringspaces=false
}

\lstdefinestyle{antlr}{
    language=Java,
    morekeywords={grammar,returns},
    keywordstyle=\\color{blue},
    basicstyle=\\ttfamily,
    tabsize=2,
    breaklines=true,
    showspaces=false,
    showstringspaces=false
}


\renewcommand{\Coloneqq}{\mathrel{\mathop{::}}=}
\newcommand*{\mybox}[1]{\framebox{#1}}

\setbeamertemplate{bibliography item}{\insertbiblabel}

% Load the actual RWTH theme. Suggested is to load the full theme,
% as it requires some specific dimensions
\usetheme{rwth}

\begin{document}

	\logo{\includegraphics{logo.png}}

% Setup presentation information
	\title{Syntaktische Mehrdeutigkeiten \\ Erkennung, Vermeidung und Auflösung}
	\date{14.06.2024}
	\author{Lennart Protte}

	\frame{\titlepage}


	\section{Motivation}\label{sec:motivation}
	\begin{frame}
		\vspace{1em}
		\begin{columns}[T]
			\column{0.5\textwidth}
			\centering
			\begin{block}{Zielsetzung}
				\vspace{1em}
				Entwicklung von Strategien zur \textbf{Erkennung} und \textbf{Lösung} von Mehrdeutigkeiten
				oder zur \textbf{Vermeidung} ihrer Entstehung.
				\vspace{1em}
			\end{block}
			\column{0.5\textwidth}
			\centering
			\begin{tikzpicture}[node distance=1.5cm, auto, every node/.style={align=center}]
				% Nodes
				\node (avoid) [rectangle, draw, left of=input, text centered, minimum height=1cm, minimum width=2.5cm, xshift=-4.5cm] {Vermeidung};
				\node (input) [rectangle, draw, text centered, minimum height=1cm, minimum width=2.5cm] {Mehrdeutigkeiten};
				\node (detect) [rectangle, draw, below of=input, text centered, minimum height=1cm, minimum width=2.5cm, yshift=-0.5cm] {Erkennung};
				\node (solve) [rectangle, draw, below of=detect, text centered, minimum height=1cm, minimum width=2.5cm, yshift=-0.5cm] {Lösung};

				% Arrows
				\draw[->] (input) -- (avoid);
				\draw[->] (input) -- (detect);
				\draw[->] (detect) -- (solve);
			\end{tikzpicture}
		\end{columns}
		\vspace{1em}
		\begin{exampleblock}{Beispiel}
			Manchmal lassen sich Mehrdeutigkeiten nicht vermeiden. \\
			Dann ist es notwendig diese zu Erkennen und Aufzulösen.
		\end{exampleblock}
	\end{frame}


	\section{Probleme und Herausforderungen}\label{sec:probleme-und-herausforderungen}
	\begin{frame}
		\begin{block}{Problemstellungen}
			\begin{itemize}
				\item Bei komplexen Programmiersprachen lassen sich Mehrdeutigkeiten kaum vermeiden.
				\item Es existiert kein Algorithmus, welcher eindeutig Mehrdeutigkeiten erkennen kann.
				\item Mehrdeutigkeiten können nur selten Algorithmisch aufgelöst werden.
			\end{itemize}
		\end{block}
	\end{frame}


	\section{Vermeidung von Mehrdeutigkeiten}\label{sec:vermeidung-von-mehrdeutigkeiten}
	\begin{frame}[fragile]
		\begin{block}{Design eindeutiger Programmiersprachen}
			\begin{itemize}
				\item Symbole
				\item Bezeichner
				\item Vorrangsregeln
				\item Assoziativität
			\end{itemize}
		\end{block}
		\begin{columns}[T,onlytextwidth]
			\column{0.45\textwidth}
			\begin{exampleblock}{Mehrdeutiger C++ Code}
				\hspace*{-3em}
				\begin{minipage}{\textwidth+3em}
					\begin{lstlisting}[style=cppStyle,label={lst:lstlisting1}]
				class A {public: void func() {}};
				class B {public: void func() {}};
				class C: public A, public B {};
				int main() {
				    C c;
				    c.func();
				    return 0;
				}
					\end{lstlisting}
				\end{minipage}
			\end{exampleblock}
			\column{0.48\textwidth}
			\begin{exampleblock}{Eindeutiger C++ Code}
				\hspace*{-3em}
				\begin{minipage}{\textwidth+3em}
					\begin{lstlisting}[style=cppStyle,label={lst:lstlisting2}]
				class A {public: void func() {}};
				class B {public: void func() {}};
				class C: public A, public B {};
				int main() {
				    C c;
				    c.A::func();
				    return 0;
				}
					\end{lstlisting}
				\end{minipage}
			\end{exampleblock}
		\end{columns}
	\end{frame}


	\section{Erkennung von Mehrdeutigkeiten}\label{sec:erkennung-von-mehrdeutigkeiten}
	\begin{frame}
		\begin{block}{Methoden zur Erkennung von Mehrdeutigkeiten}
			\begin{itemize}
				\item Suchbasierte Mehrdeutigkeitserkennung
				\item Analyse der Parsing-Tabelle
				\item Analyze der Grammatik
			\end{itemize}
		\end{block}
		\begin{exampleblock}{Suchbasierte Merhdeutigkeitserkennung}
			\includegraphics[width=\textwidth]{./detectors}\cite{springer2013}
		\end{exampleblock}
	\end{frame}


	\section{Auflösung von Mehrdeutigkeiten}\label{sec:auflsung-von-mehrdeutigkeiten}
	\begin{frame}
		\begin{columns}[T]
			\column{0.45\textwidth}
			\begin{block}{Chomsly Normalform}
				Für $A,B,C \in N$, $S \in \text{Startsymbol}$, $a \in T$ und $B,C \neq S$. \\
				\begin{align*}
					& A \rightarrow BC \\
					& A \rightarrow a \\
					& S \rightarrow \varepsilon
				\end{align*}
			\end{block}
			\begin{exampleblock}
				Einige Mehrdeutigkeiten können durch die Umformung in eine CNF aufgelöst werden.\cite{10.1007/978-3-662-21545-6_41}
				Desweiteren vereinfacht es eine weitere Analyse der Grammatik.
			\end{exampleblock}
			\column{0.45\textwidth}
			\begin{block}{Algorithmus von Vasudevan und Tratt}
				\begin{align*}
					& E \Coloneqq E * E \phantom{|} & (links) \\
					& > E + E & (links) \\
					& \phantom{|} | \phantom{|} \bold{num}
				\end{align*}
			\end{block}
			\begin{exampleblock}
				Kann unter gegebener Assotiativität und Vorrangsregeln eine eindeutige Grammatik erzeugen.

			\end{exampleblock}
		\end{columns}
	\end{frame}


	\section{Praktische Ansätze}\label{sec:praktische-ansatze}
	\begin{frame}[fragile]
			\begin{block}{ANTLR}
				\begin{minipage}{\textwidth}
					\begin{lstlisting}[style=antlr,label={lst:lstlisting3}]
						start   : statement* EOF ;
						statement: 'if' condition 'then' statement ('else' statement)?? ;

						condition: '(' STRING ')' ;
						STRING   : '"' .*? '"' ;
						WS       : [ \t\r\n]+ -> skip ;
					\end{lstlisting}
				\end{minipage}
			\end{block}
			\vspace{1em}
			\begin{exampleblock}{Erläuterung des \(??\) Operators}
				Der \(??\) Operator gibt dem Else die niedrigste Priorität, womit es mit dem äußersten If gematched wird. \\
			\end{exampleblock}\cite{parr}
	\end{frame}


	\section{Ausblick}\label{sec:ausblick-und-zukunftige-projekte}
	\begin{frame}
		\begin{itemize}
			\item Maschinelles Lernen zur Erkennung von Mehrdeutigkeiten
			\item Natürliche Sprache => Komplexe Mehrdeutige Grammatiken
		\end{itemize}
	\end{frame}


	\section{Quellen}\label{sec:quellen}
	\begin{frame}[allowframebreaks]
		\bibliographystyle{apalike}
		\bibliography{refs}
	\end{frame}

\end{document}