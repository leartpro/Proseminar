% This file provides an example Beamer presentation using the RWTH theme
% showcasing some of the more common options, similar to the Powerpoint version
% 12.11.2014: Revision 1 (Harold Bruintjes, Tim Lange)

% For RWTH, beamer should be loaded with class option t (top)
\documentclass[t]{beamer}

% Use fontspec to get Arial font
% Requires use of XeLaTeX
\usepackage{fontspec}
\setmainfont{Arial}
\setsansfont{Arial}
% Also force Arial for math for a more consistent look
\usepackage{unicode-math}

% German style date formatting (footer)
\usepackage[ddmmyyyy]{datetime}
\renewcommand{\dateseparator}{.}

\usepackage{MnSymbol,wasysym}

% Format the captions used for figures etc.
\usepackage[compatibility=false]{caption}
\captionsetup{singlelinecheck=off,justification=raggedleft,labelformat=empty,labelsep=none}

% PGFPlots is used for drawing some of the charts
\usepackage{pgfplots}
\usepackage{blkarray}
\pgfplotsset{compat=newest}
\input{plot_commands.tex}

\setbeamertemplate{bibliography item}{\insertbiblabel}
% Load the actual RWTH theme. Suggested is to load the full theme,
% as it requires some specific dimensions
\usetheme{rwth}

\begin{document}

	\logo{\includegraphics{logo.png}}

% Setup presentation information
	\title{Syntaktische Mehrdeutigkeiten \\ Erkennung, Vermeidung und Auflösung}
	\date{07.06.2024}
	\author{Lennart Protte}

	\frame{\titlepage}


	\section{Motivation}\label{sec:motivation}
	\begin{frame}
		\vspace{1em}
		\begin{columns}[T]
			\column{0.5\textwidth}
			\centering
			\begin{block}{Zielsetzung}
				\vspace{1em}
				Entwicklung von Strategien zur \textbf{Erkennung} und \textbf{Lösung} von Mehrdeutigkeiten
				oder zur \textbf{Vermeidung} ihrer Entstehung.
				\vspace{1em}
			\end{block}
			\column{0.5\textwidth}
			\centering
			\begin{tikzpicture}[node distance=1.5cm, auto, every node/.style={align=center}]
				% Nodes
				\node (avoid) [rectangle, draw, left of=input, text centered, minimum height=1cm, minimum width=2.5cm, xshift=-4.5cm] {Vermeidung};
				\node (input) [rectangle, draw, text centered, minimum height=1cm, minimum width=2.5cm] {Mehrdeutigkeiten};
				\node (detect) [rectangle, draw, below of=input, text centered, minimum height=1cm, minimum width=2.5cm, yshift=-0.5cm] {Erkennung};
				\node (solve) [rectangle, draw, below of=detect, text centered, minimum height=1cm, minimum width=2.5cm, yshift=-0.5cm] {Lösung};

				% Arrows
				\draw[->] (input) -- (avoid);
				\draw[->] (input) -- (detect);
				\draw[->] (detect) -- (solve);
			\end{tikzpicture}
		\end{columns}
		\vspace{1em}
		\begin{exampleblock}{Beispiel}
			Manchmal lassen sich Mehrdeutigkeiten nicht vermeiden. \\
			Dann ist es notwendig diese zu Erkennen und Aufzulösen.
		\end{exampleblock}
	\end{frame}


	\section{Probleme und Herausforderungen}\label{sec:probleme-und-herausforderungen}
	\begin{frame}
		\begin{block}{Problemstellungen}
			\begin{itemize}
				\item Bei komplexen Programmiersprachen lassen sich Mehrdeutigkeiten kaum vermeiden.
				\item Es existiert kein Algorithmus, welcher eindeutig Mehrdeutigkeiten erkennen kann.
				\item Mehrdeutigkeiten können nur selten Algorithmisch aufgelöst werden.
			\end{itemize}
		\end{block}
	\end{frame}


	\section{Vermeidung von Mehrdeutigkeiten}\label{sec:vermeidung-von-mehrdeutigkeiten}
	\begin{frame}
		\begin{itemize}
			\item Chomsky-NormalForm
			\item Algorithmen zur Erkennung von Mehrdeutigkeiten
			\item Lösungsstrategien: Vorrangsregeln, Assoziativität, Grammatik-Korrektur
		\end{itemize}
		\begin{center}
			\includegraphics[width=\textwidth]{./img}\cite{springer2013}
		\end{center}
	\end{frame}

	\section{Erkennung von Mehrdeutigkeiten}\label{sec:erkennung-von-mehrdeutigkeiten}
	\begin{frame}
		\begin{block}{Methoden zur Erkennung von Merhdeutigkeiten}
			\begin{itemize}
				\item Suchbasierte Mehrdeutigkeitserkennung
				\item Analyse der Parsing-Tabelle
				\item Analyze der Grammatik
			\end{itemize}
		\end{block}
		\begin{exampleblock}{Suchbasierte Merhdeutigkeitserkennung}
			\includegraphics[width=\textwidth]{./img}\cite{springer2013}
		\end{exampleblock}
	\end{frame}

	\section{Auflösung von Mehrdeutigkeiten}\label{sec:auflsung-von-mehrdeutigkeiten}
	\begin{frame}
		\begin{itemize}
			\item Chomsky-NormalForm
			\item Algorithmen zur Erkennung von Mehrdeutigkeiten
			\item Lösungsstrategien: Vorrangsregeln, Assoziativität, Grammatik-Korrektur
		\end{itemize}
		\begin{center}
			\includegraphics[width=\textwidth]{./img}\cite{springer2013}
		\end{center}
	\end{frame}


	\section{Praktische Ansätze}\label{sec:praktische-ansatze}
	\begin{frame}
		\begin{itemize}
			\item Beispiele aus der Praxis: Bekannte Compiler und Programmiersprachen
			\item Fallstudien: Anwendung der Techniken in realen Projekten , (LALR(1), LL(1) Parser)
		\end{itemize}
	\end{frame}


	\section{Ausblick}\label{sec:ausblick-und-zukunftige-projekte}
	\begin{frame}
		\begin{itemize}
			\item Maschinelles Lernen zur Erkennung von Mehrdeutigkeiten
			\item Natürliche Sprache => Komplexe MEhrdeutige Grammatiken
		\end{itemize}
	\end{frame}


	\section{Quellen}\label{sec:quellen}
	\begin{frame}[allowframebreaks]
		\bibliographystyle{apalike}
		\bibliography{refs}
	\end{frame}

\end{document}