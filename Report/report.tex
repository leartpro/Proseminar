% This is samplepaper.tex, a sample chapter demonstrating the
% LLNCS macro package for Springer Computer Science proceedings;
% Version 2.20 of 2017/10/04
%
\documentclass[runningheads]{llncs}
%
\makeatletter
\usepackage[algoruled,boxed,lined]{algorithm2e}
\makeatletter
\g@addto@macro{\@algocf@init}{\SetKwInOut{Parameter}{Parameters}}
\makeatother
\usepackage{amsmath}
\usepackage{amssymb}
\usepackage{graphicx}
% Used for displaying a sample figure. If possible, figure files should
% be included in EPS format.
%
% If you use the hyperref package, please uncomment the following line
% to display URLs in blue roman font according to Springer's eBook style:
% \renewcommand\UrlFont{\color{blue}\rmfamily}

\begin{document}
%
\title{Handhabung syntaktischer Mehrdeutigkeiten beim Parsen}
%
%\titlerunning{Abbreviated paper title}
% If the paper title is too long for the running head, you can set
% an abbreviated paper title here
%
\author{Lennart Protte\inst{1}}
%
\authorrunning{Author}
% First names are abbreviated in the running head.
% If there are more than two authors, 'et al.' is used.
%
\institute{RWTH Aachen University \email{lennart.protte@rwth-aachen.de}}
%
\maketitle              % typeset the header of the contribution
%
\begin{abstract}
Ipsum lorem ipsum dolor sit amet, consectetur adipiscing elit.
\keywords{keywords, keywords, keywords keywords}
\end{abstract}
%
%
%

%Sample citation: \cite{gelman2013bayesian}.
%See Springer website\footnote{\url{https://www.springer.com/gp/computer-science/lncs/conference-proceedings-guidelines}.} for further inforamtion on the \LaTeX style.
\section{Problem Description}

\section{State of the Art}

\section{Contribution}

\section{Conclusion}

%
% ---- Bibliography ----
%
% BibTeX users should specify bibliography style 'splncs04'.
% References will then be sorted and formatted in the correct style.
%
\bibliographystyle{splncs04}
\bibliography{refs}
\end{document}
