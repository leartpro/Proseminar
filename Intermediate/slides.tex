% This file provides an example Beamer presentation using the RWTH theme
% showcasing some of the more common options, similar to the Powerpoint version
% 12.11.2014: Revision 1 (Harold Bruintjes, Tim Lange)

% For RWTH, beamer should be loaded with class option t (top)
\documentclass[t]{beamer}

% Use fontspec to get Arial font
% Requires use of XeLaTeX
\usepackage{fontspec}
\setmainfont{Arial}
\setsansfont{Arial}
% Also force Arial for math for a more consistent look
\usepackage{unicode-math}

% German style date formatting (footer)
\usepackage[ddmmyyyy]{datetime}
\renewcommand{\dateseparator}{.}

\usepackage{MnSymbol,wasysym}

% Format the captions used for figures etc.
\usepackage[compatibility=false]{caption}
\captionsetup{singlelinecheck=off,justification=raggedleft,labelformat=empty,labelsep=none}

% PGFPlots is used for drawing some of the charts
\usepackage{pgfplots}
\pgfplotsset{compat=newest}
\input{plot_commands.tex}

%custom
\usepackage{tikz}
\usetikzlibrary{trees}
\usetikzlibrary{automata, positioning}
\usepackage{listings}
\lstset{basicstyle=\ttfamily}
\usepackage{tcolorbox}
\usepackage{framed}
\usepackage{forest}
\usepackage{xcolor}
\usepackage{mathtools}
\newcommand*{\mybox}[1]{\framebox{#1}}
\setbeamertemplate{bibliography item}{\insertbiblabel}

% Load the actual RWTH theme. Suggested is to load the full theme,
% as it requires some specific dimensions
\usetheme{rwth}

\begin{document}

    \logo{\includegraphics{logo.png}}

% Setup presentation information
    \title{Auflösen von Mehrdeutigkeiten in kontextfreien Grammatiken}
    \date{17.05.2024}
    \author{Lennart Protte}

    \frame{\titlepage}


    \section{Mehrdeutige Grammatik}
    \begin{frame}
        \begin{minipage}[c]{0.5\textwidth}
            \centering
            \begin{align*}
                & G(N,T,P,S) \\
                & N= \{E\} \\
                & T= \{+, *, \bold{num}\} \\
                & S= \{E\}  \\
                & P = \{ \\
                & E \rightarrow E + E | E * E | \bold{num} \\
                & \}
            \end{align*}
        \end{minipage}\begin{minipage}[c]{0.5\textwidth}
                          \begin{align*}
                              & E \rightarrow E + E \\
                              & \phantom{E} \rightarrow \bold{num} + E \\
                              & \phantom{E} \rightarrow \bold{num} + \bold{num} \\
                          \end{align*}
                          \begin{align*}
                              & E \rightarrow E + E \\
                              & \phantom{E} \rightarrow E + \bold{num} \\
                              & \phantom{E} \rightarrow \bold{num} + \bold{num} \\
                          \end{align*}
        \end{minipage}
        Eine Grammatik ist mehrdeutig, wenn sie mehrere Ableitungen für ein Wort zulässt.
    \end{frame}


    %\section{Operatorrangfolge}
    %\begin{frame}
    %    Die Sprache zu 'verschärfen' (vom Regelwerk her) ist eventuell nicht gewollt
    %\end{frame}


    \section{Comsky Normal Form Definition}
    \begin{frame}
        Eine kontextfreie Grammatik ist in Chomsky-Normalform, wenn jede Regel eine der folgenden Formen hat:
        \begin{align*}
            & A \rightarrow BC \\
            & A \rightarrow a \\
            & S \rightarrow \varepsilon
        \end{align*}
        wobei $A,B,C \in N$ und $a \in T$
    \end{frame}


    \section{Chonsky Normal Form Beispiel}
    %hier dann beispielhaften baum anzeichnen in präsentation
    \begin{frame}
        \begin{minipage}[c]{0.5\textwidth}
            \begin{align*}
                & G(N,T,P,S) \\
                & N= \{E\} \\
                & T= \{+, *, \bold{num}\} \\
                & S= \{E\}  \\
                & P = \{ \\
                & E \rightarrow E + E | E * E | \bold{num} \\
                & \}
            \end{align*}
        \end{minipage}\begin{minipage}[c]{0.5\textwidth}
                          \begin{align*}
                              & G(N,T,P,S) \\
                              & N= \{E\} \\
                              & T= \{+, *, \bold{num}\} \\
                              & S= \{E\}  \\
                              & P = \{ \\
                              & E \rightarrow H_0 E | H_1 E | \bold{num} \\
                              & H_0 \rightarrow E H_2 \\
                              & H_1 \rightarrow E H_3 \\
                              & H_2 \rightarrow + \\
                              & H_3 \rightarrow * \\
                              & \}
                          \end{align*}
        \end{minipage}
    \end{frame}


    \section{Operatoren mit Vorrangsregeln}
    \begin{frame}

    \end{frame}


    \section{Definition Pattern}
    \begin{frame}
        E => E + E => num + E => num + num
        ($E, \alpha, \beta, \gamma$)
    \end{frame}


    \section{Tabellen für + und *}
    \begin{frame}
        \begin{table}[h]
            \centering
            \caption{Die Semantik des > Operators in Bezug auf Muster}
            \begin{tabular}{|c|c|c|}
                \hline
                > & $E \Coloneqq E\alpha_{2}E$         \\
                \hline
                $E \Coloneqq E\alpha_{1}E$ & $(E, \bullet{E}\alpha_{1}E, {E}\alpha_{2}E)$ \\& $(E, E\alpha_{1}\bullet{E}, {E}\alpha_{2}E)$ \\
                \hline
            \end{tabular}\label{tab:table}
        \end{table}

        \begin{table}[h]
            \centering
            \caption{Die Semantik der Linksassoziativität}
            \begin{tabular}{|c|c|c|}
                \hline
                F ::= F.a.i.E                       & F ::= F.a.z.E         & \\
                \hline
                F ::= F.a.i.(e(F,e'.a.i.'F,F.a.i.E) & (F,F'.a.i.'F,F.a.z.E) & \\
                \hline
                F ::= F.a.z.(e(F,e'.a.z.'F,F.a.i.E) & (F,F'.a.z.'F,F.a.z.E) & \\
                \hline
            \end{tabular}\label{tab:table2}
        \end{table}

    \end{frame}


    \section{Anwenden von Algorithmus}
    \begin{frame}
        Hier tikz-animation package verwenden
    \end{frame}


    \section{Quellen}
    \begin{frame}[allowframebreaks]
        \bibliographystyle{apalike}
        \bibliography{refs}
    \end{frame}

\end{document}
