% This file provides an example Beamer presentation using the RWTH theme
% showcasing some of the more common options, similar to the Powerpoint version
% 12.11.2014: Revision 1 (Harold Bruintjes, Tim Lange)

% For RWTH, beamer should be loaded with class option t (top)
\documentclass[t]{beamer}

% Use fontspec to get Arial font
% Requires use of XeLaTeX
\usepackage{fontspec}
\setmainfont{Arial}
\setsansfont{Arial}
% Also force Arial for math for a more consistent look
\usepackage{unicode-math}

% German style date formatting (footer)
\usepackage[ddmmyyyy]{datetime}
\renewcommand{\dateseparator}{.}

\usepackage{MnSymbol,wasysym}

% Format the captions used for figures etc.
\usepackage[compatibility=false]{caption}
\captionsetup{singlelinecheck=off,justification=raggedleft,labelformat=empty,labelsep=none}

% PGFPlots is used for drawing some of the charts
\usepackage{pgfplots}
\pgfplotsset{compat=newest}
\input{plot_commands.tex}

%custom
\usepackage{tikz}
\usetikzlibrary{trees}
\usetikzlibrary{automata, positioning}
\usepackage{listings}
\lstset{basicstyle=\ttfamily}
\usepackage{tcolorbox}
\usepackage{framed}
\usepackage{forest}
\usepackage{xcolor}
%\usepackage{mathtools}
\renewcommand{\Coloneqq}{\mathrel{\mathop{::}}=}
\newcommand*{\mybox}[1]{\framebox{#1}}
\setbeamertemplate{bibliography item}{\insertbiblabel}

%\setbeamertemplate{caption}[numbered]

% Load the actual RWTH theme. Suggested is to load the full theme,
% as it requires some specific dimensions
\usetheme{rwth}

\begin{document}

	\logo{\includegraphics{logo.png}}

% Setup presentation information
	\title{Auflösen von Mehrdeutigkeiten in kontextfreien Grammatiken}
	\date{17.05.2024}
	\author{Lennart Protte}

	\frame{\titlepage}


	\section{Mehrdeutige Grammatik}\label{sec:mehrdeutige-grammatik}
	\begin{frame}
		\begin{minipage}[c]{0.5\textwidth}
			\centering
			\begin{align*}
				& G(N,T,P,S) \\
				& N= \{E\} \\
				& T= \{+, *, \bold{num}\} \\
				& S= \{E\}  \\
				& P = \{ \\
				& E \rightarrow E + E | E * E | \bold{num} \\
				& \}
			\end{align*}
		\end{minipage}\begin{minipage}[c]{0.5\textwidth}
			              \begin{align*}
				              & E \rightarrow E + E \\
				              & \phantom{E} \rightarrow \bold{num} + E \\
				              & \phantom{E} \rightarrow \bold{num} + \bold{num} \\
			              \end{align*}
			              \begin{align*}
				              & E \rightarrow E + E \\
				              & \phantom{E} \rightarrow E + \bold{num} \\
				              & \phantom{E} \rightarrow \bold{num} + \bold{num} \\
			              \end{align*}
		\end{minipage}
		Eine Grammatik ist mehrdeutig, wenn sie mehrere Ableitungen für ein Wort zulässt.
	\end{frame}


	%\section{Operatorrangfolge}
	%\begin{frame}
	%    Die Sprache zu 'verschärfen' (vom Regelwerk her) ist eventuell nicht gewollt
	%\end{frame}


	\section{Comsky Normal Form Definition}\label{sec:comsky-normal-form-definition}
	\begin{frame}
		Eine kontextfreie Grammatik ist in Chomsky-Normalform, wenn jede Regel eine der folgenden Formen hat:
		\begin{align*}
			& A \rightarrow BC \\
			& A \rightarrow a \\
			& S \rightarrow \varepsilon
		\end{align*}
		wobei $A,B,C \in N$ und $a \in T$
	\end{frame}


	\section{Chonsky Normal Form Beispiel}\label{sec:chonsky-normal-form-beispiel}
	%hier dann beispielhaften baum anzeichnen in präsentation
	\begin{frame}
		\begin{minipage}[c]{0.5\textwidth}
			\begin{align*}
				& G(N,T,P,S) \\
				& N= \{E\} \\
				& T= \{+, *, \bold{num}\} \\
				& S= \{E\}  \\
				& P = \{ \\
				& E \rightarrow E + E | E * E | \bold{num} \\
				& \}
			\end{align*}
		\end{minipage}\begin{minipage}[c]{0.5\textwidth}
			              \begin{align*}
				              & G(N,T,P,S) \\
				              & N= \{E\} \\
				              & T= \{+, *, \bold{num}\} \\
				              & S= \{E\}  \\
				              & P = \{ \\
				              & E \rightarrow H_0 E | H_1 E | \bold{num} \\
				              & H_0 \rightarrow E H_2 \\
				              & H_1 \rightarrow E H_3 \\
				              & H_2 \rightarrow + \\
				              & H_3 \rightarrow * \\
				              & \}
			              \end{align*}
		\end{minipage}
	\end{frame}


	\section{Operatoren mit Vorrangsregeln}\label{sec:operatoren-mit-vorrangsregeln}
	\begin{frame}

	\end{frame}


	\section{Definition von Mustern}\label{sec:muster-zur-beseitigung-der-mehrdeutigkeit-von-operatoren}
	\begin{frame}
	($S, \alpha \circ \beta, \gamma$)
		\\
		S = Das Startsymbol der Produktion \\
		$\alpha \circ \beta$ zwei Nichtterminale aus $G$ mit $\circ$ als Operator \\
		$\gamma$ als Ableitung. \\
		Ob $\alpha$ oder $\beta$ abgeleitet wird, wird durch $\bullet$ vor dem jeweiligen Nichtterminal gekennzeichnet.
		\\\\
		So kann $E => E + E => num + E => num + num$ durch
		$(E, \bullet{E}+E, num)$ und $(E, E+\bullet{E}, num)$ beschrieben werden.
	\end{frame}


	\section{Tabellen für Ordnung und Assotiativität von '+' und '*'}\label{sec:tabellen-fur-ordnung-und-assotiativitat-von-'+'-und-'*'}
	\begin{frame}
		Aus der Ordnung $ \alpha_{1} > \alpha_{2}$ und aus $\alpha_{1}, \alpha_{2}$ sind links assotiativ ergeben sich die folgenden Tabellen:
		\begin{table}[h]
			\centering
			\caption{Die Semantik der > Ordnung von Operatoren}
			\begin{tabular}{|c|c|}
				\hline
				>                          & $E \Coloneqq E\alpha_{2}E$                   \\
				\hline
				$E \Coloneqq E\alpha_{1}E$ & $(E, \bullet{E}\alpha_{1}E, {E}\alpha_{2}E)$ \\
				& $(E, E\alpha_{1}\bullet{E}, E\alpha_{2}E)$   \\
				\hline
			\end{tabular}\label{tab:table}
		\end{table}

		\begin{table}[h]
			\centering
			\caption{Die Semantik der Linksassoziativität}
			\begin{tabular}{|c|c|c|}
				\hline
				links                      & $E \Coloneqq E\alpha_{1}E$                 & $E \Coloneqq E\alpha_{2}E$                 \\
				\hline
				$E \Coloneqq E\alpha_{1}E$ & $(E, E\alpha_{1}\bullet{E}, E\alpha_{1}E)$ & $(E, E\alpha_{1}\bullet{E}, E\alpha_{2}E)$ \\
				\hline
				$E \Coloneqq E\alpha_{2}E$ & $(E, E\alpha_{2}\bullet{E}, E\alpha_{1}E)$ & $(E, E\alpha_{2}\bullet{E}, E\alpha_{2}E)$ \\
				\hline
			\end{tabular}\label{tab:table2}
		\end{table}
	\end{frame}


	\section{Beispiel Algorithmus}\label{sec:beispiel-algorithmus}
	\begin{frame}
		$E \Coloneqq E * E (links) > E + E | num $
		\begin{tabular}{|c|c|c|}
			\hline
			Operator & Vorrang & Assoziativität   \\
			\hline
			*        & 2       & links assoziativ \\
			\hline
			+        & 1       & links assoziativ \\
			\hline
		\end{tabular}\label{tab:table3}
		Ergeben die Muster zur Beseitigung von Mehrdeutigkeiten:
		\begin{align*}
		(E, E*\bullet{E}, E*E)
		(E, \bullet{E}*E, E+E)
		(E, E+\bullet{E}, E*E)
		(E, \bullet{E}+E, E+E)
		\end{align*}
		Grammatik erhält vier neue Nichtterminale
		% kopieren der Ursprungsgrammatik mit nachfolgendem anwenden der Pattern
		\begin{align*}
			& G(N,T,P,S) N= \{E, E_1, E_2, E_3, E_4\} T= \{+, *, \bold{num}\} S= \{E\} \\
			& P = \{ \\
			& E     \rightarrow E_1 * E_2 | E_3 + E_4 | \bold{num} \\
			& E_1   \rightarrow E_1 * E_2 | \bold{num} \\
			& E_2   \rightarrow E_3 + E_4 | \bold{num} \\
			& E_3   \rightarrow E_1 * E_2 | \bold{num} \\
			& E_4   \rightarrow E_3 + E_4 | \bold{num} \\
			& \}
		\end{align*}
	\end{frame}


	\section{Beispiel Algorithmus}\label{sec:beispiel-algorithmus2}
	\begin{frame}
		\begin{minipage}[c]{0.5\textwidth}
			\begin{align*}
				& G(N,T,P,S) \\
				& N= \{E, E_1, E_2, E_3, E_4\} \\
				& T= \{+, *, \bold{num}\} \\
				& S= \{E\} \\
				& P = \{ \\
				& E     \rightarrow E_1 * E_2 | E_3 + E_4 | \bold{num} \\
				& E_1   \rightarrow E_1 * E_2 | \bold{num} \\
				& E_2   \rightarrow E_3 + E_4 | \bold{num} \\
				& E_3   \rightarrow E_1 * E_2 | \bold{num} \\
				& E_4   \rightarrow E_3 + E_4 | \bold{num} \\
				& \}
			\end{align*}
		\end{minipage}\begin{minipage}[c]{0.5\textwidth}
			              \begin{align*}
				              & G(N,T,P,S) \\
				              & N= \{E, E_1, E_2, E_3, E_4\} \\
				              & T= \{+, *, \bold{num}\} \\
				              & S= \{E\} \\
				              & P = \{ \\
				              & E     \rightarrow E_1 * E_2 | E_3 + E_4 | \bold{num} \\
				              & E_1   \rightarrow E_1 * E_5 | \bold{num} \\
				              & E_2   \rightarrow E_3 + E_4 | \bold{num} \\
				              & E_3   \rightarrow E_5 * E_2 | \bold{num} \\
				              & E_4   \rightarrow E_3 + E_4 | \bold{num} \\
				              & E_5   \bold{num} \\
				              & \}
			              \end{align*}
		\end{minipage}
	\end{frame}


	\section{Quellen}\label{sec:quellen}
	\begin{frame}[allowframebreaks]
		\bibliographystyle{apalike}
		\bibliography{refs}
	\end{frame}

\end{document}
