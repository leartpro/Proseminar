% This file provides an example Beamer presentation using the RWTH theme
% showcasing some of the more common options, similar to the Powerpoint version
% 12.11.2014: Revision 1 (Harold Bruintjes, Tim Lange)

% For RWTH, beamer should be loaded with class option t (top)
\documentclass[t]{beamer}

% Use fontspec to get Arial font
% Requires use of XeLaTeX
\usepackage{fontspec}
\setmainfont{Arial}
\setsansfont{Arial}
% Also force Arial for math for a more consistent look
\usepackage{unicode-math}

% German style date formatting (footer)
\usepackage[ddmmyyyy]{datetime}
\renewcommand{\dateseparator}{.}

\usepackage{MnSymbol,wasysym}

% Format the captions used for figures etc.
\usepackage[compatibility=false]{caption}
\captionsetup{singlelinecheck=off,justification=raggedleft,labelformat=empty,labelsep=none}

% PGFPlots is used for drawing some of the charts
\usepackage{pgfplots}
\pgfplotsset{compat=newest}
\input{plot_commands.tex}

%custom
\usepackage{tikz}
\usetikzlibrary{trees}
\usetikzlibrary{automata, positioning}
\usepackage{listings}
\lstset{basicstyle=\ttfamily}
\usepackage{tcolorbox}
\usepackage{framed}
\usepackage{forest}
\usepackage{xcolor}
\newcommand*{\mybox}[1]{\framebox{#1}}
\setbeamertemplate{bibliography item}{\insertbiblabel}



% Load the actual RWTH theme. Suggested is to load the full theme,
% as it requires some specific dimensions
\usetheme{rwth}

\begin{document}

\logo{\includegraphics{logo.png}}

% Setup presentation information
\title{Syntaktische Mehrdeutigkeiten}
\date{20.04.2024}
\author{Lennart Protte}
\frame{\titlepage}

\section{Grammatiken}
\begin{frame}[fragile]
\begin{align*}
    & G(N,T,P,S) \\
    & N= \{zuweisung, var, op, ausdr, num\} \\
    & T= \{=, +, *, \bold{bez}, \bold{num}\} \\
    & S= \{zuweisung\}  \\
    & P = \{ \\
    & zuweisung \rightarrow var = ausdr \\
    & var \rightarrow \bold{bez} \\
    & num \rightarrow \bold{num} \\
    & ausdr \rightarrow var \mid num \mid ausdr + ausdr \mid ausdr * ausdr \\
    & \}
\end{align*}
\begin{center}
\vspace{\fill}
\begin{lstlisting}[frame=single,escapechar=|]
    |\mybox{in} \mybox{=} \mybox{3}|
    |\mybox{res} \mybox{=} \mybox{2} \mybox{*} \mybox{in} \mybox{+} \mybox{7}|
\end{lstlisting}
\end{center}
\end{frame}

\section{Parse Baum}
\begin{frame}
\begin{center}
\begin{forest}
  [program
    [zuweisung
      [var
        [\mybox{in}]
      ]
      [op
        [{\mybox{=}}, content format={\forestoption{content}}]
      ]
      [ausdr
        [num
            [\mybox{3}]
        ]
      ]
    ]
    [zuweisung
      [var
        [\mybox{res}]
      ]
      [op
        [{\mybox{=}}, content format={\forestoption{content}}]
      ]
      [\textcolor{blue}{ausdr}
        [\textcolor{blue}{ausdr}
            [\textcolor{blue}{num}
                [\textcolor{blue}{\mybox{2}}]
            ]
            [\textcolor{blue}{op}
                [\textcolor{blue}{\mybox{*}}]
            ]
            [\textcolor{blue}{var}
                [\textcolor{blue}{\mybox{in}}]
            ]
        ]
        [\textcolor{blue}{op}
            [\textcolor{blue}{\mybox{+}}]
        ]
        [\textcolor{blue}{num}
            [\textcolor{blue}{\mybox{7}}]
        ]
      ]
    ]
  ]
\end{forest}
\end{center}
\end{frame}

\section{Uneindeutigkeiten}
\begin{frame}[fragile]
\begin{center}
\begin{column}{0.5\textwidth}
\begin{center}
\begin{forest}
      [ausdr
        [ausdr
            [num
                [\mybox{2}]
            ]
            [op
                [\mybox{*}]
            ]
            [var
                [\mybox{in}]
            ]
        ]
        [op
            [\mybox{+}]
        ]
        [num
            [\mybox{7}]
        ]
      ]
\end{forest}
\end{center}
\end{column}
\begin{column}{0.5\textwidth}
\begin{center}
\begin{forest}
      [ausdr
      [num
            [\mybox{2}]
        ]
        [op
            [\mybox{*}]
        ]
        [ausdr
            [var
                [\mybox{in}]
            ]
            [op
                [\mybox{+}]
            ]
            [num
                [\mybox{7}]
            ]
        ]
      ]
\end{forest}
\end{center}
\end{column}
\end{center}
\end{frame}

\section{Motivation}
\begin{frame}
\begin{center}
        \vspace{20pt}
        \begin{itemize}
        \item \textbf{Bedeutung}: \par
        Warum ist es wichtig, Doppeldeutigkeiten in Grammatiken zu vermeiden?\cite{watrous}  \par
        
        \vspace{10pt}

        \item \textbf{Erkennung}: \par
        Wie kann man uneindeutige Grammatiken erkennen?\cite{springer2013} 
        
        \vspace{10pt}
        
        \item \textbf{Problemstellung}: \par
        Wie kann man Uneindeutigkeiten in Grammatiken lösen?\cite{wharton1976} 
        
        \vspace{10pt}
        
        \item \textbf{Herausforderung}: \par
        Es gibt keine generelle Lösung für dieses Problem!\cite{thorup1994} 
    \end{itemize}
\end{center}
\end{frame}

% Frame with items
\section{Aktueller Forschungsstand}
\begin{frame}
\centering
\vspace{20pt}
    \begin{itemize}
        \setlength\itemsep{1em} % Abstand zwischen den Hauptelementen
        \item \textbf{praktische Ansätze}:
        \begin{itemize}
            \setlength\itemsep{0.8em} % Abstand zwischen den Subelementen
            \item Vorrangsregeln \cite{thorup1994}
            \item Assotiativität
            \item Gruppierung
            \item Korrektur der Grammatik \cite{watrous2020}
        \end{itemize}
        
        \vspace{10pt}
        
        \item \textbf{Probleme}:
        \begin{itemize}
            \setlength\itemsep{0.8em} % Abstand zwischen den Subelementen
            \item Syntaktische Mehrdeutigkeiten \par
            "Dangling Else"-Problem
            \item Lexikalische Mehrdeutigkeiten \par
            "+"-Operator
        \end{itemize}
    \end{itemize}
\end{frame}


\section{Quellen}
\begin{frame}[allowframebreaks]
\bibliographystyle{apalike}
\bibliography{refs}
\end{frame}
\end{document}